\documentclass[12pt]{exam}

\newcommand{\ds}{\ensuremath{\displaystyle}}

\usepackage{amsmath,amsfonts, amsthm}
\usepackage{multicol}
\usepackage{multirow}
\usepackage{harpoon}
\renewcommand{\arraystretch}{1.5}

\newcommand{\harpvec}[1]{\overrightharp{\ensuremath{\mathbf{#1}}}}
\newcommand{\vect}[1]{\ensuremath{\mathbf{#1}}}
\newcommand{\<}{\langle}
\renewcommand{\>}{\rangle}
\newcommand{\p}{\partial}

% ref: http://pgfplots.sourceforge.net/gallery.html
% ref: http://tex.stackexchange.com/a/74575/79754
\usepackage{pgfplots}% This uses tikz
\pgfplotsset{compat=newest}% use newest version
\tikzset{LineStyle/.style={smooth, ultra thick, samples=400}}

% \printanswers

\begin{document}

\begin{center}
\fbox{\fbox{\parbox{5.5in}{\centering
MATH 3142-001 | Spring 2016 | Dr. Clontz | Final Exam Part 1
}}}
\end{center}
\vspace{0.1in}
\makebox[\textwidth]{
  Name:\enspace\hrulefill\hrulefill\hrulefill
}

\vspace{12pt}

\begin{itemize}
  \item Each problem is labeled with its worth toward the total grade
        of 100 points for this final exam.
  \item You do not need to show your work on these multiple-choice problems. No
        partial credit will be given.
  \item You may not use any notes/electronics on this portion of the exam.
  \item This part of the midterm is due after 30 minutes. Materials submitted
        late will be penalized by \(50\%\).
\end{itemize}

\newpage

\noindent
For each of the following statements, choose if it is True or False.
\bigskip
\begin{questions}

\question[1]
A continuous function on a closed bounded interval is integrable.
\begin{choices}
\item True
\item False
\end{choices}
\bigskip
\question[1]
Let \(A\subseteq\mathbb R^n\) and \(F:A\to\mathbb R^m\) be continuous.
If \(F(A)\) is sequentially compact, then \(A\) is sequentially compact.
\begin{choices}
\item True
\item False
\end{choices}
\bigskip
\question[1]
The sequence of points \(\{\frac{1}{n}\}\) in \(\mathbb R\) is Cauchy.
\begin{choices}
\item True
\item False
\end{choices}
\bigskip
\question[1]
Every function \(f:\mathbb R^n\to\mathbb R\)
with first-order partial derivatives is continuous.
\begin{choices}
\item True
\item False
\end{choices}
\bigskip
\question[1]
The exponential function \(\operatorname{exp}:\mathbb R\to\mathbb R\)
is the unique function whose derivative is itself.
\begin{choices}
\item True
\item False
\end{choices}

\newpage

Choose the most appropriate response for each.

\bigskip

\question[1]
Which of these is a closed subset of \(\mathbb R\)?
\begin{choices}
\item \(\{\frac{1}{2^n}:n\in\mathbb N\}\)
\item \(\{0\}\cup\{\frac{1}{n+1}:n\in\mathbb N\}\)
\item \(\mathbb Q\)
\end{choices}
\bigskip
\question[1]
Which of these is a sequentially compact subset of \(\mathbb R^2\)?
\begin{choices}
\item \(\{(x,y)\in\mathbb R^2: x^2+y^2<4\}\)
\item \(\{(x,y)\in\mathbb R^2: y=3x-4\}\)
\item \(\{(x,y)\in\mathbb R^2: \max(|x|,|y|)=1\}\)
\end{choices}
\bigskip
\question[1]
Which of these is not a requirement for a metric \(d:X^2\to[0,\infty)\)?
\begin{choices}
\item \(d(x,y)=d(y,x)\) for all \(x,y\in X\)
\item \(d(x,y)d(y,z)d(x,z)=1\) for all \(x,y,z\in X\)
\item \(d(x,z)\leq d(x,y)+d(y,z)\) for all \(x,y,z\in X\)
\end{choices}
\bigskip
\question[1]
Let \(f:\mathbb R^2\to\mathbb R\).
Which of these is equal to \(\ds\lim_{t\to0}\frac{f(x+t,y+2t)-f(x,y)}{t}\)?
\begin{choices}
\item \(\<(1,2),\nabla f(x,y)\>\)
\item \(f_x(x,y)+2f_y(x,y)\)
\item \(\frac{\partial^2 f}{\partial x\partial y}(x,2y)\)
\end{choices}
\bigskip
\question[1]
Which of these functions \(f:\mathbb R\to\mathbb R\) satisfies
\(f(0)=0\), \(f'(0)=1\), \(f''(x)=-f(x)\)?
\begin{choices}
\item \(f(x)=\sin x\)
\item \(f(x)=\cos x\)
\item \(f(x)=\ln x\)
\end{choices}
\end{questions}


\newpage

\begin{center}
\fbox{\fbox{\parbox{5.5in}{\centering
MATH 3142-001 | Spring 2016 | Dr. Clontz | Final Exam Part 2
}}}
\end{center}
\vspace{0.1in}
\makebox[\textwidth]{
  Name:\enspace\hrulefill\hrulefill\hrulefill
}

\vspace{12pt}

\begin{itemize}
  \item Each problem is labeled with its worth toward the total grade
        of 100 points for this midterm.
  \item You must choose four of the eight problems to submit. These should
        be stapled to this cover sheet. Save the other four problems
        for your reference in Part 3.
  \item (Approved students may defer an extra problem to Part 3.)
  \item Each problem requires a rigorous proof. When in doubt, don't skip
        details. You may sketch pictures to help illustrate concepts, but
        the proof must still be valid without the use of illustrations.
  \item You may use your notes or textbook once Part 1 has been submitted.
        Electronics are still disallowed.
  \item This part of the midterm is due after 150 minutes. Materials submitted
        late will be penalized by \(50\%\).
\end{itemize}

\newpage

\newpage

\begin{center}
\fbox{\fbox{\parbox{5.5in}{\centering
MATH 3142-001 | Spring 2016 | Dr. Clontz | Final Exam Part 3
}}}
\end{center}
\vspace{0.1in}
\makebox[\textwidth]{
  Name:\enspace\hrulefill\hrulefill\hrulefill
}

\vspace{12pt}

\begin{itemize}
  \item Each problem is labeled with its worth toward the total grade
        of 100 points for this midterm.
  \item You must choose two of the four problems you didn't choose
        for Part 2 to submit by email.
  \item (Approved students who deferred a problem to Part 3 should submit
        three total by email.)
  \item Each problem requires a rigorous proof. When in doubt, don't skip
        details. You may sketch pictures to help illustrate concepts, but
        the proof must still be valid without the use of illustrations.
  \item You may use any notes you wish, and even collaborate with other
        students, as long as you submit your own work.
        \textbf{Plagiarism will be treated as a violation of academic honesty.}
  \item Solutions must be typeset using \LaTeX{}. The professor will
        provide a template via email.
  \item This part of the midterm is due by email at 11:59pm on Friday, March 13.
        Materials submitted late will not be graded.
\end{itemize}

\newpage

\begin{questions}
\question[15]
Define \(f:[0,2]\to\mathbb R\) by \(f(x)=3x\). Explicitly define a sequence of
partitions \(\{P_n\}\) of \([0,2]\), and then prove that this sequence
is Archimedian for \(f\) on \([0,2]\).

\newpage
\question[15]
Compute the boundary of the subset \(\mathbb Q^2\) of \(\mathbb R^2\).

\newpage
\question[15]
Prove that the function \(f:\mathbb R^2\to\mathbb R\) defined by
\[
f(x,y) =
\begin{cases}
  \frac{x^2y}{x^3+y^3} & (x,y)\not=(0,0) \\
  0 & (x,y)=(0,0)
\end{cases}
\]
is not continuous.

\newpage
\question[15]
Let \(f:\mathbb R^n\to\mathbb R\) be defined by
\[
f(\vect x) =
\begin{cases}
  \|\vect x\| & \|\vect x\|\in \mathbb R\setminus\mathbb Q\\
  0 & \|\vect x\|\in\mathbb Q
\end{cases}
.\]
Prove that \(f\) is continuous at \(\vect 0\).

\newpage
\question[15]
Let \(d:X^2\to[0,\infty)\) be a metric on \(X\). Prove that the function
\(e:X^2\to[0,\infty)\) defined by \(e(x,y)=\min(d(x,y),1)\) satisfies
the triangle inequality for the below cases:

\textbf{Case \(e(x,y)+e(y,z)<1\):}
\vfill
\vfill

\textbf{Case \(e(x,y)+e(y,z)\geq 1\):}
\vfill


\newpage
\question[15]
Compute \(f_y(0,0)\) where \(f:\mathbb R^2\to\mathbb R\) is defined by
\[
f(x,y) =
\begin{cases}
  \frac{x^2y}{x^3+y^3} & (x,y)\not=(0,0) \\
  0 & (x,y)=(0,0)
\end{cases}
.\]

\newpage
\question[15]
Let \(f:\mathbb R^n\to\mathbb R\) be continuous such that
\(f(\vect 0)=0\). Prove that for each vector \(\vect p\in\mathbb R^n\),
there exists another vector \(\vect x\in\mathbb R^n\) such that
\(\frac{\partial f}{\partial\vect p}(\vect x)=f(\vect p)\).

\newpage
\question[15]
Recall that \(\ln:(0,\infty)\to\mathbb R\) is defined
by \(\ln(x)=\int_1^x\frac{1}{t}dt\).
Show that \(f:(-\infty,0)\to\mathbb R\) defined by
\(f(x)=\ln(-x)\) is the unique solution to the differential equation
\[
\begin{cases}
  f'(x)=\frac{1}{x} & \forall x\in(-\infty,0) \\
  f(-1)=0
\end{cases}
.\]

\end{questions}

\end{document}
