\documentclass[letterpaper, twoside, 12pt]{book}
\usepackage{notes}

\title{MATH 3142 Notes | Spring 2016}
\date{Updated: \today}
\author{Your Name Here\\ UNC Charlotte}

\begin{document}

\maketitle

This document is a template for you to take notes in my MATH 3142 course.
For your note check grade, you are required to complete all proofs/solutions
for the problems specified. This template will be updated periodically
throughout the course; you are responsible for updating your copy
as the template is updated. See the syllabus for more details.

You should maintain your notes on Overleaf.com and provide me with a
link so I can check on them. I'll give you notice before notes are ``due'';
when they are due I will download a copy myself from Overleaf.

This is not a replacement for the textbook for this course,
\textit{Advanced Calculus} by Patrick M. Fitzpatrick. Many proofs are
outlined in that text, as well as all the relevant definitions and other
results not included in these notes.

A proof is valid if and only if it uses concepts proven previously in
the book. For example, you cannot prove a lemma in Chapter 6 using
a theorem from Chapter 10, but using a proposition from Chapter 4
is allowed.

I hope you enjoy working through these results. Please email me with
any questions.

\noindent| Dr. Steven Clontz \(\<\)sclontz5@uncc.edu\(\>\)
















\setcounter{chapter}{5}
\chapter{Integration: Two Fundamental Theorems}







\section{Darboux Sums: Upper and Lower Integrals}

\begin{lemma}[6.1]
  Suppose that the function \(f:[a,b]\to\mb R\) is bounded and the numbers
  \(m,M\) have the property that
  \[
    m\leq f(x)\leq M
  \]
  for all \(x\) in \([a,b]\). Then, if \(P\) is a partition of the domain
  \([a,b]\),
  \[
    m(b-a)\leq L(f,P)
      \text{ and }
    U(f,P)\leq M(b-a)
  .\]
\end{lemma}
\begin{proof}

\end{proof}


\begin{lemma}[6.2, The Refinement Lemma]
  Suppose that the function \(f:[a,b]\to\mb R\) is bounded and that \(P\)
  is a partition of its domain \([a,b]\). If \(P^\star\) is a refinement
  of \(P\), then
  \[
    L(f,P)\leq L(f,P^\star)
      \text{ and }
    U(f,P^\star)\leq U(f,P)
  .\]
\end{lemma}
\begin{proof}

\end{proof}


\begin{lemma}[6.3]
  Suppose that the function \(f:[a,b]\to\mb R\) is bounded and that
  \(P_1,P_2\) are partitions of its domain. Then \(L(f,P_1)\leq U(f,P_2)\).
\end{lemma}
\begin{proof}

\end{proof}


\begin{lemma}[6.4]
  For a bounded function \(f:[a,b]\to\mb R\),
  \[
    \underline{\int_a^b} f
      \leq
    \overline{\int_a^b} f
  .\]
\end{lemma}
\begin{proof}

\end{proof}

\begin{exercise}[2]
  For an interval \([a,b]\) and a positive number \(\delta\),
  show that there is a partition \(P=\{x_i:0\leq i\leq n\}\) of
  \([a,b]\) such that each partition interval \([x_i,x_{i+1}]\)
  of \(P\) has length less than \(\delta\).
\end{exercise}
\begin{solution}

\end{solution}


\begin{exercise}[3]
  Suppose that the bounded function \(f:[a,b]\to\mb R\) has the property
  that for each rational number \(x\) in the interval \([a,b]\),
  \(f(x)=0\). Prove that
  \[
    \underline{\int_a^b}f
      \leq
    0
      \leq
    \overline{\int_a^b}f
  .\]
\end{exercise}
\begin{solution}

\end{solution}


\begin{exercise}[6]
  Suppose that \(f:[a,b]\to\mb R\) is a bounded function for which there is
  a partition \(P\) of \([a,b]\) with \(L(f,P)=U(f,P)\). Prove that
  \(f:[a,b]\to\mb R\) is constant.
\end{exercise}
\begin{solution}

\end{solution}




\section{The Archimedes-Riemann Theorem}


\begin{lemma}[6.7]
  For a bounded function \(f:[a,b]\to\mb R\) and a partition \(P\) of
  \([a,b]\),
  \[
    L(f,P)
      \leq
    \underline{\int_a^b}f\leq\overline{\int_a^b}f\leq U(f,P)
  .\]
\end{lemma}
\begin{proof}

\end{proof}


\begin{theorem}[6.8, The Archimedes-Riemann Theorem]
  Let \(f:[a,b]\to\mb R\) be a bounded function. Then \(f\) is integrable on
  \([a,b]\) if and only if there is a sequence \(\{P_n\}\) of partitions
  of the interval \([a,b]\) such that
  \[
    \lim_{n\to\infty}[U(f,P_n)-L(f,P_n)]=0
  .\]
  Moreover, for any such sequence of partitions,
  \[
    \lim_{n\to\infty} L(f,P_n)
      =
    \int_a^b f
      =
    \lim_{n\to\infty} U(f,P_n)
  .\]
\end{theorem}
\begin{proof}

\end{proof}


\begin{example}[6.9]
  Show that
  a monotonically increasing function \(f:[a,b]\to\mb R\) is integrable.
\end{example}
\begin{solution}

\end{solution}

\begin{example}[6.11]
  Show that \(\int_0^1 x^2\,dx=\frac{1}{3}\).
\end{example}
\begin{solution}

\end{solution}


\begin{exercise}[4]
  Prove that for a natural number \(n\),
  \[
    \sum_{i=1}^n i = \frac{n(n+1)}{2}
  .\]
  Then use this fact and the Archimedes-Riemann Theorem to show that
  \(\int_a^b x\,dx=(b^2-a^2)/2\).
\end{exercise}
\begin{solution}

\end{solution}


\begin{exercise}[6b]
  Use the Archimedes-Riemann Theorem to show that for \(0\leq a<b\),
  \[
    \int_a^b x^2\,dx = \frac{b^3-a^3}{3}
  .\]
\end{exercise}
\begin{solution}

\end{solution}


\begin{exercise}[9]
  Suppose that the functions \(f:[a,b]\to\mb R\) and
  \(g:[a,b]\to\mb R\) are integrable. Show that there is a sequence
  \(\{P_n\}\) of partitions of \([a,b]\) that is an Archimediean sequence
  of partitions for \(f\) on \([a,b]\) and also an Archimedean sequence
  of partitions for \(g\) on \([a,b]\).
\end{exercise}
\begin{solution}

\end{solution}




\section{Additivity, Monotonicity, and Linearity}


\begin{theorem}[6.12, Additivity over Intervals]
  Let \(f:[a,b]\to\mb R\) be integrable on \([a,b]\) and let \(c\in(a,b)\).
  Then \(f\) is integrable on \([a,c]\) and \([c,b]\), and furthermore
  \[
    \int_a^b f = \int_a^c f + \int_c^b f
  .\]
\end{theorem}
\begin{proof}

\end{proof}


\begin{theorem}[6.13, Monotonicity of the Integral]
  Suppose \(f,g:[a,b]\to\mb R\) are integrable and that \(f(x)\leq g(x)\)
  for all \(x\in[a,b]\). Then
  \[
    \int_a^b f \leq \int_a^b g
  .\]
\end{theorem}
\begin{proof}

\end{proof}


\begin{lemma}[6.14]
  Let \(f,g:[a,b]\to\mb R\) be bounded and let \(P\) partition \([a,b]\).
  Then
  \[
    L(f,P)+L(g,P)\leq L(f+g,P)
      \text{~~and~~}
    U(f+g,P)\leq U(f,P)+U(g,P)
  .\]
  Moreover, for any number \(\alpha\),
  \[
    U(\alpha f,P)=\alpha U(f,P)
      \text{~~and~~}
    L(\alpha f,P)=\alpha L(f,P)
      \text{~~if~}
    \alpha\geq 0
  \]
  \[
    U(\alpha f,P)=\alpha L(f,P)
      \text{~~and~~}
    L(\alpha f,P)=\alpha U(f,P)
      \text{~~if~}
    \alpha< 0
  .\]
\end{lemma}
\begin{proof}

\end{proof}


\begin{theorem}[6.15, Linearity of the Integral]
  Let \(f,g:[a,b]\to\mb R\) be integrable. Then for any two numbers
  \(\alpha,\beta\), the function \(\alpha f+\beta g:[a,b]\to\mb R\) is
  integrable and
  \[
    \int_a^b[\alpha f+\beta g]=\alpha\int_a^b f + \beta\int_a^b g
  .\]
\end{theorem}
\begin{proof}

\end{proof}


\begin{exercise}[1]
  Suppose that the functions \(f,g,f^2,g^2,fg\) are integrable on \([a,b]\).
  Prove that \((f-g)^2\) is also integrable on \([a,b]\) and that
  \(\int_a^b(f-g)^2\geq0\). Use this to prove that
  \[
    \int_a^b fg
      \leq
    \frac{1}{2}\left[
      \int_a^b f^2 + \int_a^b g^2
    \right]
  .\]
\end{exercise}
\begin{solution}

\end{solution}

\begin{exercise}[4]
  Suppose that \(S\) is a nonempty bounded set of numbers and that \(\alpha\)
  is a number. Define \(\alpha S\) to be the set \(\{\alpha x:x\in S\}\).
  Prove that
  \[
    \sup\alpha S=\alpha\sup S
      \text{~~and~~}
    \inf\alpha S=\alpha\inf S
      \text{~~if~}
    \alpha\geq 0
  \]
  while
  \[
    \sup\alpha S=\alpha\inf S
      \text{~~and~~}
    \inf\alpha S=\alpha\sup S
      \text{~~if~}
    \alpha< 0
  .\]
\end{exercise}
\begin{solution}

\end{solution}


\begin{exercise}[6]
  Suppose that \(f:[a,b]\to\mb R\) is bounded and let \(a<c<b\). Prove that if
  \(f\) is integrable on both \([a,c],[c,b]\), then it is integrable on
  \([a,b]\).
\end{exercise}
\begin{solution}

\end{solution}




\section{Continuity and Integrability}


\begin{lemma}[6.17]
  Let the function \(f:[a,b]\to\mb R\) be continuous let \(P\) partition
  its domain. Then there is a partition interval of \(P\) that contains two
  points \(u,v\) for which the following estimate holds:
  \[
    0
      \leq
    U(f,P)-L(f,P)
      \leq
    [f(v)-f(u)][b-a]
  .\]
\end{lemma}
\begin{proof}

\end{proof}


\begin{theorem}[6.18]
  A continuous function on a closed bounded interval is integrable.
\end{theorem}
\begin{proof}

\end{proof}


\begin{theorem}[6.19]
  Supose \(f:[a,b]\to\mb R\) is bounded on \([a,b]\) and continuous on
  \((a,b)\). Then \(f\) is integrable on \([a,b]\) and the value of
  \(\int_a^b f\) does not depend on the values of \(f\) at the endpoints
  of \([a,b]\).
\end{theorem}
\begin{proof}

\end{proof}

\begin{exercise}[1]
  Determine whether each of the following statements is true or false, and
  justify your answer.
  \begin{enumerate}[(a)]
    \item If \(f:[a,b]\to\mb R\) is integrable and \(\int_a^b f=0\), then
      \(f(x)=0\) for all \(x\in[a,b]\).
    \item If \(f:[a,b]\to\mb R\) is integrable, then \(f\) is continuous.
    \item If \(f:[a,b]\to\mb R\) is integrable and \(f(x)\geq0\) for all
      \(x\in[a,b]\), then \(\int_a^b f\geq 0\).
    \item A continuous function \(f:(a,b)\to\mb R\) defined on an open interval
      \((a,b)\) is bounded.
    \item A continuous function \(f:[a,b]\to\mb R\) defined on a closed interval
      \([a,b]\) is bounded.
  \end{enumerate}
\end{exercise}
\begin{solution}
  \begin{enumerate}[(a)]
    \item
    \item
    \item
    \item
    \item
  \end{enumerate}
\end{solution}


\begin{exercise}[5]
  Suppose that the continuous function \(f:[a,b]\to\mb R\) has the property
  \[
    \int_c^d f\leq 0
      \text{~~whenever~}
    a\leq c<d\leq b
  .\]
  Prove that \(f(x)\leq 0\) for all \(x\in[a,b]\). Is this true if we only
  require integrability of the function?
\end{exercise}
\begin{solution}

\end{solution}


\begin{exercise}[6]
  Suppose that \(f:[0,1]\to\mb R\) is continuous and that \(f(x)\geq 0\) for
  all \(x\in[0,1]\). Prove that \(\int_0^1 f>0\) if and only if there is a
  point \(x_0\in[0,1]\) at which \(f(x_0)>0\).
\end{exercise}
\begin{solution}

\end{solution}




\section{The First Fundamental Theorem: Integrating Derivatives}


\begin{lemma}[6.21]
  Suppose \(f:[a,b]\to\mb R\) is integrable and that the number \(A\) has
  the property that for every \(P\) partitioning \([a,b]\),
  \[
    L(f,P) \leq A \leq U(f,P)
  .\]
  Then
  \[
    \int_a^b f = A
  .\]
\end{lemma}
\begin{proof}

\end{proof}


\begin{theorem}[6.22, The First Fundamental Theorem: Integrating Derivatives]
  Let \(F:[a,b]\to\mb R\) be continuous on \([a,b]\) and differentiable on
  \((a,b)\). Moreover, suppose that its derivative
  \(F':(a,b)\to\mb R\) is both continuous and bounded. Then
  \[
    \int_a^b F'(x)~dx
      =
    F(b)-F(a)
  .\]
\end{theorem}
\begin{proof}

\end{proof}


\begin{exercise}[1]
  Let \(m,b\) be positive numbers. Find the value of \(\int_0^1 mx+b ~dx\)
  in the following three ways:
  \begin{enumerate}[(a)]
    \item Using elementary geometry, interpreting the integral as an area.
    \item Using upper and lower Darboux sums based on regular partitions of
      the interval \([0,1]\) and using the Archimedes-Riemann Theorem.
    \item Using the First Fundamental Theorem (Integrating Derivatives).
  \end{enumerate}
\end{exercise}
\begin{solution}

\end{solution}


\begin{exercise}[5]
  The monotonicity property of the integral implies that if the functions
  \(g,h:[0,\infty)\to\mb R\) are continuous and \(g(x)\leq h(x)\) for all
  \(x\geq 0\), then
  \[
    \int_0^x g\leq \int_0^x h
    \text{~~ for all~} x\geq 0
  .\]
  Use this and the First Fundamental Theorem to show that each of the following
  inequalities implies the next:
  \[
    \cos x \leq 1
    \text{~~ if~} x\geq 0
  .\]
  \[
    \sin x \leq x
    \text{~~ if~} x\geq 0
  .\]
  \[
    1-\cos x \leq \frac{x^2}{2}
    \text{~~ if~} x\geq 0
  .\]
  \[
    x-\sin x \leq \frac{x^3}{6}
    \text{~~ if~} x\geq 0
  .\]
  \[
    x-\frac{x^3}{6} \leq \sin x \leq x
    \text{~~ if~} x\geq 0
  .\]

  (For this problem, you may assume that the sine and cosine functions
  are differentiable functions with the properties
  \(\sin(0)=0\), \(\cos(0)=1\), \(\frac{d}{dx}[\sin(x)]=\cos(x)\),
  and \(\frac{d}{dx}[\cos(x)]=-\sin(x)\).)
\end{exercise}
\begin{solution}

\end{solution}




\section{The Second Fundamental Theorem: Differentiating Integrals}


\begin{theorem}[6.26, The Mean Value Theorem for Integrals]
  Suppose that \(f:[a,b]\to\mb R\) is continuous. Then there is a point \(x_0\)
  in the interval \([a,b]\) at which
  \[
    \frac{1}{b-a}\int_a^b f
      =
    f(x_0)
  .\]
\end{theorem}
\begin{proof}

\end{proof}


\begin{proposition}[6.27]
  Suppose that the function \(f:[a,b]\to\mb R\) is integrable. Define
  \[
    F(x) = \int_a^x f
    \text{~~for all~} x\in[a,b]
  .\]
  Then the function \(F:[a,b]\to\mb R\) is continuous.
\end{proposition}
\begin{proof}

\end{proof}


\begin{theorem}[6.29, The Second Fundamental Theorem: Differentiating Integrals]
  Suppose that \(f:[a,b]\to\mb R\) is continuous. Then
  \[
    \frac{d}{dx}\left[\int_a^x\right]
      =
    f(x)
    \text{~~for all~} x\in(a,b)
  .\]
\end{theorem}
\begin{proof}

\end{proof}


\begin{exercise}[2b]
  Suppose \(f:[0,2]\to\mb R\) is defined by
  \[
    f(x) =
    \begin{cases}
      x^2 & \text{if } 0\leq x\leq 1 \\
      x   & \text{if } 1< x\leq 2
    \end{cases}
  .\]
  Define
  \[
    F(x)=\int_a^x f(t)~dt
    \text{~~for all} x\in[a,b]
  \]
  and find a formula for \(F(x)\) which does not involve integrals.
\end{exercise}
\begin{solution}

\end{solution}


\begin{exercise}[5]
  Suppose \(f:\mb R\to\mb R\) is continuous. Define
  \[
    G(x)
      =
    \int_0^x (x-t)f(t)~dt
    \text{~~for all~} x
  .\]
  Prove that \(G''(x)=f(x)\) for all \(x\).
\end{exercise}
\begin{solution}

\end{solution}


\begin{exercise}[12]
  Suppose that \(f,g:[a,b]\to\mb R\) are continuous and that \(\alpha,\beta\)
  are real numbers. Define
  \[
    H(x)
      =
    \int_a^x[\alpha f+\beta g]-\alpha\int_a^x[f]-\beta\int_a^x[g]
    \text{~~for all~} x\in[a,b]
  .\]
  Prove that \(H(a)=0\) and \(H'(x)=0\) for all \(x\in(a,b)\).
  Use this fact and the Identity Criterion to give an alternate proof of
  Theorem 6.15 for continuous functions.
\end{exercise}
\begin{solution}

\end{solution}



\setcounter{chapter}{9}
\chapter{The Euclidean Space \texorpdfstring{$\mb R^n$}{Rn}}


\section{The Linear Structure of \texorpdfstring{$\mb R^n$}{Rn}
and the Scalar Product}

\begin{proposition}[10.2]
  Let \(\vect u,\vect v,\vect w\in\mb R^n\)
  and \(\alpha,\beta\in\mb R\). Then both of the following hold:
  \[
    \<\vect u,\vect v\>=\<\vect v,\vect u\>
  \]
  \[
    \<\alpha\vect u+\beta\vect w,v\>
      =
    \alpha\<\vect u,\vect v\>+\beta\<\vect w,\vect v\>
  \]
\end{proposition}
\begin{proof}

\end{proof}

\begin{lemma}[10.4]
  For \(\vect u,\vect v\in\mb R^n\), \(\vect u,\vect v\) are
  orthogonal if and only if
  \(\|\vect u+\vect v\|=\|\vect u\|^2+\|\vect v\|^2\).
\end{lemma}
\begin{proof}

\end{proof}

\begin{lemma}[10.5]
  For \(\vect u,\vect v\in\mb R^n\) where \(\vect v\not=\vect 0\),
  define \(\lambda=\frac{\<\vect u,\vect v\>}{\<\vect v,\vect v\>}\)
  and \(\vect w=\vect u-\lambda\vect v\). Then \(\vect v,\vect w\)
  are orthogonal and \(\vect u=\vect w+\lambda\vect v\).
\end{lemma}
\begin{proof}

\end{proof}

\begin{theorem}[10.6, The Cauchy-Schwarz Inequality]
  For any two vectors \(\vect u,\vect v\in\mb R^n\),
  \[
    |\<\vect u,\vect v\>|
      \leq
    \|\vect u\|\|\vect v\|
  .\]
\end{theorem}
\begin{proof}

\end{proof}

\begin{theorem}[10.7, The Triangle Inequality]
  For any two vectors \(\vect u,\vect v\in\mb R^n\),
  \[
    \|\vect u+\vect v\|
      \leq
    \|\vect u\|+\|\vect v\|
  .\]
\end{theorem}

\begin{exercise}[3]
  Show that for \(\vect u\in\mb R^n\), \(\alpha\in\mb R\):
  \begin{enumerate}[(a)]
    \item \(\|\vect u\|=0\) if and only if \(\vect u=\vect 0\).
    \item \(\|\alpha\vect u\|=|\alpha|\|\vect u\|\).
  \end{enumerate}
\end{exercise}
\begin{proof}

\end{proof}

\begin{exercise}[4]
  For \(\vect u,\vect v\in\mathbb R^n\) verify the identity
  \[
    \|\vect u-\vect v\|^2
      =
    \|\vect u\|^2+\|\vect v\|^2-2\<\vect u,\vect v\>
  .\]
\end{exercise}
\begin{solution}

\end{solution}

\begin{exercise}[9]
  Let \(\vect u\in\mb R^n\) and suppose \(\|\vect u\|<1\).
  Show that for \(\vect v\in\mb R^n\),
  \(\|\vect v-\vect u\|<1-\|\vect u\|\) implies
  \(\|\vect v\|<1\).
\end{exercise}
\begin{solution}

\end{solution}

\begin{exercise}[10]
  Let \(\vect u\in\mb R^n\) and \(r>0\). Suppose \(\vect v,\vect w\in\mb R^n\)
  are at a distance less than \(r\) from \(\vect u\). Prove that if
  \(0\leq t\leq 1\), then the point \(t\vect v+(1-t)]\vect w\) is also
  at a distance less than \(r\) from \(\vect u\).
\end{exercise}
\begin{solution}

\end{solution}


\section{Convergence of Sequences in \texorpdfstring{$\mb R^n$}{Rn}}

\begin{theorem}[10.9, The Componentwise Convergence Criterion]
  Let \(\{\vect u_k\}\) be a sequence in \(\mb R^n\). Then
  \(\{\vect u_k\}\) converges to \(\vect u\) if and only if
  \(\{p_i(\vect u_k)\}\) converges to \(p_i(\vect u)\) for
  all \(1\leq i\leq n\).
\end{theorem}
\begin{proof}

\end{proof}

\begin{theorem}[10.10]
  Let \(\{\vect u_k\}\), \(\{\vect v_k\}\) be sequences in \(\mb R^n\)
  such that \(\{\vect u_k\}\) converges to \(\vect u\) and
  \(\{\vect v_k\}\) converges to \(\vect v\). Then for any
  \(\alpha,\beta\in\mb R\),
  \[
    \lim_{k\to\infty}[\alpha\vect u_k+\beta\vect v_k]
      =
    \alpha\vect u+\beta\vect v
  .\]
\end{theorem}
\begin{proof}

\end{proof}

\begin{exercise}[1]
  Let \(\{\vect u_k\}\) be a sequence in \(\mb R^n\) that converges to
  \(\vect u\). Prove the following for all \(\vect v\in\mb R^n\):
  \[
    \lim_{k\to\infty}\<\vect u_k,\vect v\>=\<\vect u,\vect v\>
  .\]
\end{exercise}
\begin{solution}

\end{solution}

\begin{exercise}[2]
  Let \(\{\vect u_k\}\) be a sequence in \(\mb R^n\) and
  \(\vect u\in\mb R^n\). Prove that if
  \[
    \lim_{k\to\infty}\<\vect u_k,\vect v\>=\<\vect u,\vect v\>
  \]
  holds for all \(\vect v\in\mb R^n\), then \(\{\vect u_k\}\) converges
  to \(\vect u\).
\end{exercise}
\begin{solution}

\end{solution}

\begin{exercise}[5]
  Let \(\{\vect u_k\}\) be a sequence in \(\mb R^n\) that converges to
  \(\vect u\) where \(\|\vect u\|=r>0\). Prove that there is an index \(K\)
  where
  \[
    \|\vect u_k\|>\frac{r}{2}
    \text{~~if~}
    k\geq K
  .\]
\end{exercise}
\begin{solution}

\end{solution}


\section{Open Sets and Closed Sets in \texorpdfstring{$\mb R^n$}{Rn}}


\begin{example}[10.11]
  Let \(a<b\) be in \(\mb R\). Then \(\interior(a,b]=(a,b)\).
\end{example}
\begin{proof}

\end{proof}

\begin{example}[10.12]
  Let \(\mb Q\subseteq\mb R\) be the set of rational real numbers.
  Then \(\interior\mb Q=\emptyset\).
\end{example}
\begin{proof}

\end{proof}

\begin{proposition}[10.13]
  Every open ball \(B_r(\vect u)\) in \(\mb R^n\) is open.
\end{proposition}
\begin{proof}

\end{proof}

\begin{example}[10.14]
  Let \(a<b\) be in \(\mb R\). Then \([a,b]\) is closed.
\end{example}
\begin{proof}

\end{proof}

\begin{example}[10.15]
  The set
  \[
    [-1,1]\times[-1,1]
      =
    \{(x,y)\in\mb R^2:-1\leq x\leq 1 \text{~and~} -1\leq y\leq 1\}
  \]
  is closed in \(\mb R^2\).
\end{example}
\begin{proof}

\end{proof}

\begin{theorem}[10.16, The Complementing Characterization]
  A subset \(A\subseteq\mb R^n\) is open if and only if its complement
  \(\mb R^n\setminus A\) is closed.
\end{theorem}
\begin{proof}

\end{proof}

\begin{proposition}[10.17.i]
  The union of a collection of open subsets of \(\mb R^n\) is open.
\end{proposition}
\begin{proof}

\end{proof}

\begin{proposition}[10.17.ii]
  The intersection of a collection of closed subsets of \(\mb R^n\) is closed.
\end{proposition}
\begin{proof}

\end{proof}

\begin{proposition}[10.18.i]
  The intersection of a finite collection of
  open subsets of \(\mb R^n\) is open.
\end{proposition}
\begin{proof}

\end{proof}

\begin{proposition}[10.18.ii]
  The union of a finite collection of
  closed subsets of \(\mb R^n\) is closed.
\end{proposition}
\begin{proof}

\end{proof}

\begin{proposition}[10.19.i]
  \(A\subseteq\mb R^n\) is open if and only if \(A\cap\boundary A=\emptyset\).
\end{proposition}
\begin{proof}

\end{proof}

\begin{proposition}[10.19.ii]
  \(A\subseteq\mb R^n\) is closed if and only if \(\boundary A\subseteq A\).
\end{proposition}
\begin{proof}

\end{proof}

\begin{exercise}[2]
  Determine which of the following subsets of \(\mb R^2\) are open, closed,
  neither, or both.
  \begin{enumerate}[(a)]
    \item \(\{(x,y):x^2>y\}\)
    \item \(\{(x,y):x^2+y^2=1\}\)
    \item \(\{(x,y):x \text{~is rational}\}\)
    \item \(\{(x,y):x\geq 0,~y\geq0\}\)
  \end{enumerate}
\end{exercise}
\begin{solution}
  \begin{enumerate}[(a)]
    \item
    \item
    \item
    \item
  \end{enumerate}
\end{solution}

\begin{exercise}[3]
  Let \(r>0\) and \(O=\{\vect u\in\mb R^n:\|\vect u\|>r\}\). Prove that \(O\)
  is open.
\end{exercise}
\begin{solution}

\end{solution}

\begin{exercise}[7a]
  Show that \(A\subseteq\mb R^n\) is open if and only if
  \[
    \vect w + A = \{\vect w+\vect u:\vect u\in A\}
  \]
  is open for all \(\vect w\in\mb R^n\).
\end{exercise}
\begin{solution}

\end{solution}

\begin{exercise}[12]
  For \(A\subseteq\mb R^n\), denote its closure by
  \[
    \closure A = \interior A \cup \boundary A
  .\]
  Prove that \(A\subseteq \closure A\). Then prove that
  \(A=\closure A\) if and only if \(A\) is closed.
\end{exercise}
\begin{solution}

\end{solution}





\chapter{Continuity, Compactness, and Connectedness}


\section{Continuous Functions and Mappings}

\begin{proposition}[11.1]
  For each \(i\in\{1,\dots,n\}\), the \(i\)th projection map
  \(p_i:\mb R^n\to\mb R\) is continuous.
\end{proposition}
\begin{proof}

\end{proof}

\begin{theorem}[11.3]
  Let \(\vect u\in A\subseteq \mb R^n\) and \(h,g:A\to\mb R\) be continuous
  at \(\vect u\). Then for \(\alpha,\beta\in\mb R\), the following functions
  are continuous at \(\vect u\):
  \[
    \alpha h+\beta g : A\to\mb R
      \hspace{4em}
    h\cdot g : A\to\mb R
  .\]
  Also if \(g(\vect v)\not=0\) for all \(\vect v\in A\), then the following
  function is also continuous at \(\vect u\):
  \[
    \frac{h}{g} : A\to\mb R
  .\]
\end{theorem}
\begin{proof}

\end{proof}

\begin{theorem}[11.5]
  Let \(\vect u\in A\subseteq \mb R^n\) and \(G: A\to\mb R^m\) be continuous
  at \(\vect u\). Also let \(G(A)\subseteq B\subseteq \mb R^m\) and
  \(H:B\to\mb R^k\) be continuous at \(G(\vect u)\). Then the composition
  \[
    H\circ G : A\to\mb R^k
  \]
  is continuous at \(\vect u\).
\end{theorem}
\begin{proof}

\end{proof}

\begin{theorem}[11.9, The Componentwise Continuity Criterion]
  Let \(\vect u\in A\subseteq \mb R^n\) and \(F:A\to\mb R^m\).
  Then \(F\) is continuous at \(\vect u\) if and only if
  \(F_i=p_i\circ F:A\to\mb R\) is continuous at \(\vect u\) for each
  \(i\in\{1,\dots,n\}\).
\end{theorem}
\begin{proof}

\end{proof}

\begin{theorem}[11.11, Exercise 12]
  Let \(\vect u\in A\subseteq \mb R^n\) and \(F:A\to\mb R^m\).
  Then \(F\) is continuous at \(\vect u\) if and only if for each
  \(\epsilon>0\) there exists \(\delta>0\) such that
  \(\|\vect v-\vect u\|<\delta\)
  implies \(\|F(\vect v)-F(\vect u)\|<\epsilon\).
\end{theorem}
\begin{proof}

\end{proof}

\begin{theorem}[11.12]
  Let \(U\subseteq\mb R^n\) be open and \(F:U\to\mb R^m\).
  Then \(F\) is continuous
  if and only if \(F^{-1}(V)\) is an open subset of \(\mb R^n\) for every
  open \(V\subseteq\mb R^m\).
\end{theorem}
\begin{proof}

\end{proof}

\begin{example}[11.15]
  Use corollary 11.13 and proposition 10.18.i to prove that
  \(U=\{\vect u\in\mb R^n:a<\|\vect u\|<b\}\) is open.
  (You may assume \(f(\vect u)=\|\vect u\|\) is continuous.)
\end{example}
\begin{solution}

\end{solution}

\begin{exercise}[3]
  Fix a point \(\vect v\in\mb R^n\). Prove that \(f:\mb R^n\to\mb R\)
  defined by \(f(\vect u)=\<\vect u,\vect v\>\) is continuous.
\end{exercise}
\begin{solution}

\end{solution}

\begin{exercise}[6]
  Suppose \(f,g:\mb R^n\to\mb R\) are continuous. Prove that
  \(\{\vect u\in\mb R^n:f(\vect u)=g(\vect u)=0\}\) is closed.
  (Hint: use corollary 11.13 and proposition 10.17.ii.)
\end{exercise}
\begin{solution}

\end{solution}

\begin{exercise}[11]
  Let \(A\subseteq\mb R^n\). The characteristic function
  \(\phi_A:\mb R^n\to\mb R\) for \(A\) is defined to be
  \[
    \phi_A(\vect u) =
    \begin{cases}
      1 & \text{if } \vect u\in A \\
      0 & \text{if } \vect u\not\in A
    \end{cases}
  .\]
  Prove that \(\phi_A\) is continuous at points in \(\interior A\)
  and \(\exterior A\), but not continuous at points in \(\boundary A\).
\end{exercise}
\begin{solution}

\end{solution}


\section{Sequential Compactness, Extreme Values, and Uniform Continuity}

\begin{theorem}[11.16]
  Every sequentially compact subset of \(\mb R^n\) is bounded and closed.
\end{theorem}
\begin{proof}

\end{proof}

\begin{theorem}[11.17]
  Every bounded sequence in \(\mb R^n\) has a convergent subsequence.
\end{theorem}
\begin{proof}

\end{proof}

\begin{theorem}[11.18, The Sequential Compactness Theorem]
  A subset of \(\mb R^n\) is sequentially compact if and only if
  it is closed and bounded.
\end{theorem}
\begin{proof}

\end{proof}

\begin{corollary}[11.19]
  The generalized rectangle
  \(\prod_{i=1}^n[a_i,b_i]=[a_1,b_1]\times\dots\times[a_n,b_n]\subseteq\mb R^n\)
  is sequentially compact.
\end{corollary}
\begin{proof}

\end{proof}

\begin{theorem}[11.20]
  Let \(A\subseteq\mb R^n\) and suppose \(F:A\to\mb R^m\) is continuous.
  If \(A\) is sequentially compact then \(F(A)\) is also sequentially
  compact.
\end{theorem}
\begin{proof}

\end{proof}

\begin{lemma}[11.21]
  Every nonempty seqeuntially compact subset of \(\mb R\) has a
  maximum and minimum element.
\end{lemma}
\begin{proof}

\end{proof}

\begin{theorem}[11.22, The Extreme Value Theorem, Bolzano-Weierstrass Theorem]
  Let \(A\subseteq\mb R^n\) be nonempty sequentially compact
  and suppose \(f:A\to\mb R\) is continuous.
  Then \(f\) attains a smallest and largest value.
\end{theorem}
\begin{proof}

\end{proof}

\begin{theorem}[11.24]
  Let \(A\subseteq\mb R^n\) be nonempty.
  \(A\) is sequentially compact if and only if
  every continuous \(f:A\to\mb R\) attains a smallest and largest value
  (i.e. it has the Extreme Value Property).
\end{theorem}
\begin{proof}

\end{proof}

\begin{theorem}[11.25, Exercise 5]
  Let \(A\subseteq\mb R^n\) and \(f:A\to\mb R^m\) be continuous.
  If \(A\) is sequentially compact then \(f\) is uniformly continuous.
\end{theorem}
\begin{proof}

\end{proof}

\begin{theorem}[11.27, Exercise 11]
  Let \(A\subseteq\mb R^n\) and \(f:A\to\mb R^m\).
  \(f\) is uniformly continuous if and only if for all \(\epsilon>0\)
  there exists \(\delta>0\) such that \(\|\vect u-\vect v\|<\delta\)
  implies \(\|f(\vect u)-f(\vect v)\|<\epsilon\).
\end{theorem}
\begin{proof}

\end{proof}

\begin{exercise}[3,4]
  Recall that \(B_r(\vect u)=\{\vect v:\|\vect u-\vect v\|<r\}\).
  Is \(B_r(\vect u)\) bounded? Sequentially compact?
\end{exercise}
\begin{solution}

\end{solution}

\begin{exercise}[2]
  Let \(D_r(\vect u)=\{\vect v:\|\vect u-\vect v\|\leq r\}\).
  Prove \(D_r(\vect u)\) is sequentially compact.
\end{exercise}
\begin{solution}

\end{solution}

% \begin{exercise}[optional, Heine-Borel Theorem]
%   Let \(A\subseteq\mb R^n\). We say \(A\) is compact when for every
%   collection \(\mathcal U\) such that \(U\) is open
%   for all \(U\in\mathcal U\) and \(A\subseteq\bigcup\mathcal U\), there exists
%   a finite subset \(\mathcal F\subseteq\mathcal U\) such that
%   \(A\subseteq\bigcup\mathcal F\). Prove that \(A\) is compact if and only if
%   \(A\) is closed and bounded.
% \end{exercise}
% \begin{solution}

% \end{solution}




\chapter{Metric Spaces}

\begin{definition}
  A pair \((X,d)\) is called a \textit{metric space} if \(X\) is a set
  and \(d\) is a function \(d:X^2\to[0,\infty)\) satisfying the following
  properties:
  \begin{itemize}
    \item Identity: \(d(p,q)=0\) if and only if \(p=q\).
    \item Symmetry: \(d(p,q)=d(q,p)\).
    \item Triangle Inequality: \(d(p,q)\leq d(p,w)+d(w,q)\).
  \end{itemize}
\end{definition}

\begin{theorem}[12.2]
  \(dist(\vect p,\vect q)=\|\vect q-\vect p\|\) is
  a metric on \(\mathbb R^n\).
\end{theorem}
\begin{proof}

\end{proof}

\begin{definition}
  Let \((X,d)\) be a metric space. For \(p\in X,r>0\),
    \[ B_r(p)=\{q\in X:d(p,q)<r\} \]
  is the open ball about \(p\) with radius \(r\). For \(A\subseteq X\),
  \begin{itemize}
    \item \(\interior A = \{q\in A:\exists r>0(B_r(q)\subseteq A)\}\)
    \item \(\exterior A = \{q\in A:\exists r>0(B_r(q)\subseteq X\setminus A)\}\)
    \item \(\boundary A = \{q\in A:\forall r>0(B_r(q)\cap A\not=\emptyset\text{ and }B_r(q)\setminus A\not=\emptyset)\}\)
  \end{itemize}
  Call \(A\) open in \((X,d)\) if \(A=\interior A\).
  Note that these concepts match the definitions we gave for \(\mb R^n\)
  using the metric \(d(\vect p,\vect q)=\|\vect q-\vect p\|\).
\end{definition}

\begin{theorem}[12.8]
  Let \((X,d)\) be a metric space. Let \(p\in X,r>0\). Then
  \(B_r(p)\) is open.
\end{theorem}
\begin{proof}

\end{proof}

\begin{definition}
  Let \(d\) be a metric on \(\mb R^n\). We say \(d\) is \textit{compatible
  with the usual topology on \(\mb R^n\)} if the open sets determined by
  \(d\) are exactly the open sets determined by \(dist\).
\end{definition}


\begin{example}
  \(s:\mb R^n\to[0,\infty)\) defined by
  \(s(\vect u,\vect v)=\max\{|p_i(\vect v)-p_i(\vect u)|:1\leq i\leq n\}\)
  is a metric on \(\mb R^n\).
\end{example}
\begin{proof}

\end{proof}

\begin{theorem}
  \(s\) is compatible with the usual topology on \(\mb R^n\).
\end{theorem}
\begin{proof}

\end{proof}


\begin{example}
  \(t:\mb R^n\to[0,\infty)\) defined by
  \(t(\vect u,\vect v)=\sum_{i=1}^n |p_i(\vect v)-p_i(\vect u)|\)
  is a metric on \(\mb R^n\).
\end{example}
\begin{proof}

\end{proof}

\begin{theorem}
  \(t\) is compatible with the usual topology on \(\mb R^n\).
\end{theorem}
\begin{proof}

\end{proof}

\begin{definition}
  \(d:X\to[0,\infty)\) defined by
  \(d(p,q)=1\) for \(p\not=q\)
  and \(d(p,p)=0\)
  is called a \textit{discrete metric} on \(X\).
\end{definition}


\begin{theorem}[12.4]
  The discrete metric on \(X\) is a metric.
\end{theorem}
\begin{proof}

\end{proof}

\begin{theorem}
  The discrete metric on \(\mb R^n\) is not compatible with the usual
  topology on \(\mb R^n\). (Hint: show that every subset of a discrete
  metric space is open.)
\end{theorem}
\begin{proof}

\end{proof}

\begin{definition}
  Let \(C([a,b],\mb R)\) be the set of all continuous functions
  \(f:[a,b]\to\mb R\), and for \(f,g\in C([a,b],\mb R)\) let
  \(d(f,g)=\max\{|f(x)-g(x)|:x\in[a,b]\}\).
\end{definition}

\begin{theorem}[12.3]
  \(d(f,g)=\max\{|f(x)-g(x)|:x\in[a,b]\}\) is a metric on \(C([a,b],\mb R)\).
\end{theorem}
\begin{proof}

\end{proof}

\begin{definition}
  Let \(\{p_k\}\) denote a \textit{sequence} in a metric space \((X,d)\),
  i.e. a function from \(\mb N\) to \(X\).
\end{definition}

\begin{definition}
  We say the sequence \(\{p_k\}\) \textit{converges} to \(p\in X\)
  when \[\lim_{k\to\infty}d(p_k,p)=0.\]
\end{definition}

\begin{definition}
  \(C\subseteq X\) is said to be \textit{closed} in the metric space
  \((X,d)\) when for every sequence \(\{p_k\}\) of points in \(C\)
  converging to \(p\in X\), it follows that \(p\in C\).
\end{definition}

\begin{example}[12.11]
  The set \(\{f\in C([a,b],\mb R):f(x)\geq 0\}\) is closed.
\end{example}
\begin{proof}

\end{proof}

\begin{theorem}[12.12, The Complementing Characterization]
  Let \((X,d)\) be a metric space and \(A\subseteq X\). Then \(A\) is
  open in \((X,d)\) if and only if \(X\setminus A\) is closed in \((X,d)\).
\end{theorem}
\begin{proof}

\end{proof}

\begin{definition}
  A sequence \(\{p_k\}\) in a metric space \((X,d)\) is called a
  \textit{Cauchy sequence} when for each \(\epsilon>0\) there is an
  \(N\in\mb N\) such that \(k,l\geq N\) implies \(d(p_k,p_l)<\epsilon\).
\end{definition}

\begin{proposition}[12.15]
  Every convergent sequence in a metric space is Cauchy.
\end{proposition}
\begin{proof}

\end{proof}

\begin{lemma}[9.3]
  Every Cauchy sequence in \((\mb R,dist)\) is bounded.
\end{lemma}
\begin{proof}

\end{proof}

\begin{theorem}[9.4]
  A sequence in \((\mb R,dist)\) is Cauchy if and only if it is convergent.
\end{theorem}
\begin{proof}

\end{proof}

\begin{corollary}[Example 12.16]
  A sequence in \((\mb R^n,dist)\) is Cauchy if and only if it is convergent.
\end{corollary}
\begin{proof}

\end{proof}

\begin{definition}
  A \textit{complete metric space} is a metric space where every
  Cauchy sequence is convergent.
\end{definition}




\chapter{Differentiating Functions of Several Variables}


\section{Limits}

\begin{definition}
  Let \(A\subseteq\mb R^n\). We call \(\vect x_*\in\mb R^n\) a
  \textit{limit point} of \(A\) in the case that there exists a sequence
  in \(A\setminus\{\vect x_*\}\) which converges to \(\vect x_*\).
\end{definition}

\begin{definition}
  Let \(A\subseteq\mb R^n\) have a limit point \(x_*\in\mb R\), and
  \(f:A\to\mb R\) be a function. Then we say the
  \textit{limit of \(f\) as \(\vect x\) approaches \(\vect x_*\)}
  is \(L\in\mb R\), or
  \[
    \lim_{\vect x\to\vect x_*} f(\vect x)
      =
    L
  \]
  in the case that whenever \(\{\vect x_k\}\) is a sequence of points in
  \(A\setminus\{\vect x_*\}\) converging to \(\vect x_*\),
  then \(\{f(\vect x_k)\}\) is a sequence of real numbers which
  converges to \(L\).
\end{definition}

\begin{theorem}[13.3]
  Let\(A\subseteq\mb R^n\) and \(\vect x_*\) be a limit point of \(A\).
  Suppose the functions \(f,g:A\to\mb R\) satsify
  \[
    \lim_{\vect x\to\vect x_*} f(\vect x)
      =
    L_1
    ~~~\text{and}~~~
    \lim_{\vect x\to\vect x_*} g(\vect x)
      =
    L_2
  .\]
  Then
  \[
    \lim_{\vect x\to\vect x_*} [f(\vect x)+g(\vect x)]
      =
    L_1+L_2
  \]
  and
  \[
    \lim_{\vect x\to\vect x_*} [f(\vect x)g(\vect x)]
      =
    L_1L_2
  .\]
  And assuming \(g(\vect x)\not=0\) for \(x\in A\) and \(L_2\not=0\),
  \[
    \lim_{\vect x\to\vect x_*} [f(\vect x)/g(\vect x)]
      =
    L_1/L_2
  .\]
\end{theorem}
\begin{proof}

\end{proof}

\begin{example}[13.4]
  The limit
  \[
    \lim_{(x,y)\to(0,0)}\frac{xy}{x^2+y^2}
  \]
  does not exist.
\end{example}
\begin{proof}

\end{proof}

\begin{example}[13.5]
  \[
    \lim_{(x,y)\to(0,0)}\frac{x^3}{x^2+y^2}
      =
    0
  .\]
\end{example}
\begin{proof}

\end{proof}

\begin{exercise}[4]
  Let \(m,n\in\mb N\). Prove that
  \[
    \lim_{(x,y)\to(0,0)}\frac{x^ny^m}{x^2+y^2}
  \]
  exists if and only if \(m+n>2\).
\end{exercise}
\begin{solution}

\end{solution}

\begin{exercise}[5]
  Give an example of a subset \(A\subseteq\mb R\) and point \(x\in A\)
  such that \(x\) is not a limit point of \(A\).
\end{exercise}
\begin{solution}

\end{solution}

\begin{exercise}[12]
  Show that \(A\subseteq\mb R^n\) is closed if and only if it contains
  all its limit points.
\end{exercise}
\begin{solution}

\end{solution}


\section{Partial Derivatives}

\begin{definition}
  For each \(1\leq i\leq n\), let \(\vect e_i\in\mb R^n\)
  satisfy \(p_i(\vect e_i)=1\) and \(p_j(\vect e_i)=0\) for \(j\not=i\).
\end{definition}

\begin{definition}
  Let \(\vect x\in U\subseteq\mb R^n\) with \(U\) open. For a function
  \(f:U\to\mb R\), define its \textit{first-order
  partial derivative with respect to
  its \(i\)th component at \(\vect x\)} to be
  \[
    \left[\frac{\partial}{\partial x_i}f\right](\vect x)
      =
    \frac{\partial f}{\partial x_i}(\vect x)
      =
    f_{x_i}(\vect x)
      =
    \lim_{t\to 0}\frac{f(\vect x+t\vect e_i)-f(\vect x)}{t}
  \]
  whenever the limit exists.
\end{definition}

\begin{definition}
  Let \(U\subseteq\mb R^n\) be open. For a function
  \(f:U\to\mb R\) such that \(f_{x_i}(\vect x)\) exists for all \(\vect x\in U\),
  let \(f_{x_i}:U\to\mb R^n\) be defined as its
  \textit{first-order partial derivative with respect to
  its \(i\)th component}.
\end{definition}

\begin{definition}
  Let \(U\subseteq\mb R^n\) be open. A function
  \(f:U\to\mb R\) such that \(f_{x_i}\) exists for all \(1\leq i\leq n\) is said
  to have \textit{first-order partial derivatives}.
\end{definition}

\begin{example}[13.8*]
  If \(f:\mb R^3\to\mb R\) is defined by
  \[
    f(x,y,z)=xyz-3xy^2
  \]
  then \(f_y:\mb R^3\to\mb R\) satisfies
  \[
    f_y(x,y,z)=xz-6xy
  .\]
\end{example}
\begin{proof}

\end{proof}

\begin{example}[13.9]
  The function \(f:\mb R^2\to\mb R\) defined by
  \[
    f(x,y) = \begin{cases}
      xy/(x^2+y^2) & \text{if } (x,y)\not=(0,0) \\
      0 & \text{if } (x,y)=(0,0) \\
    \end{cases}
  \]
  has first-order partial derivatives, but is not continuous.
\end{example}
\begin{proof}

\end{proof}

\begin{definition}
  Let \(U\subseteq\mb R^n\) be open. Then a function
  \(f:U\to\mb R\) is \textit{continuously differentiable} provided that
  \(f_{x_i}:U\to\mb R\) exists and is continuous for \(1\leq i\leq n\).
\end{definition}

\begin{definition}
  Let \(U\subseteq\mb R^n\) be open, and \(f:U\to\mb R^n\) have first-order
  partial derivatives. Then for \(1\leq i,j\leq n\) let
  \[
    \frac{\partial^2 f}{\partial x_j\partial x_i}
      :
    U \to \mb R
  \]
  be the partial derivative of \(\partial f/\partial x_i:U\to\mb R\)
  with respect to its \(j\)th component. This is also denoted by
  \(f_{x_ix_j}\). When \(i=j\), this is also denoted by
  \(\frac{\partial^2 f}{\partial x_i^2}\).
\end{definition}

\begin{definition}
  Let \(U\subseteq\mb R^n\) be open. A function
  \(f:U\to\mb R\) such that \(f_{x_ix_j}\) exists for all
  \(1\leq i,j\leq n\) is said
  to have \textit{second-order partial derivatives}.
\end{definition}

\begin{definition}
  Let \(U\subseteq\mb R^n\) be open. A function
  \(f:U\to\mb R\) such that \(f_{x_ix_j}\) exists and is continuous for all
  \(1\leq i,j\leq n\) is said
  to have \textit{continuous second-order partial derivatives}.
\end{definition}

\begin{lemma}[13.11]
  Let \(U\subseteq \mb R^2\) be open and nonempty, and suppose
  \(f:U\to\mb R\) has second-order partial derivatives. Then there are
  points \((x_1,y_1),(x_2,y_2)\in U\) such that
  \(f_{xy}(x_1,y_1)=f_{yx}(x_2,y_2)\).
\end{lemma}
\begin{proof}

\end{proof}

\begin{theorem}[13.10]
  Let \(U\subseteq \mb R^n\) be open and nonempty, and suppose
  \(f:U\to\mb R\) has continuous second-order partial derivatives.
  Then for all \(1\leq i,j\leq n\), it follows that
  \(f_{x_ix_j}=f_{x_jx_i}\).
\end{theorem}
\begin{proof}[Proof for n=2]

\end{proof}

\begin{example}[13.12, exercise 13]
  The function \(f:\mb R^2\to\mb R\) defined by
  \[
    f(x,y) = \begin{cases}
      xy(x^2-y^2)/(x^2+y^2) & \text{if } (x,y)\not=(0,0) \\
      0 & \text{if } (x,y)=(0,0) \\
    \end{cases}
  \]
  has second-order partial derivatives, but
  \[
    f_{xy}(0,0)=-1
      ~~~\text{while}~~~
    f_{yx}(0,0)=1
  .\]
\end{example}
\begin{proof}

\end{proof}

\begin{exercise}[4]
  Prove that \(g:\mb R^2\to\mb R\) satisfying \(|g(x,y)|\leq x^2+y^2\)
  must have partial derivatives with respect to both \(x\) and \(y\)
  at the point \((0,0)\).
\end{exercise}
\begin{solution}

\end{solution}


\section{The Mean Value Theorem and Directional Derivatives}

\begin{lemma}[13.14, The Mean Value Lemma]
  Let \(U\subseteq\mb R^n\) be open and \(1\leq i\leq n\). Let
  \(f:U\to\mb R\) be a function with a partial derivative with respect to its
  \(i\)th component at each point in \(U\). Let \(\vect x\in U\) and
  \(a\in\mb R\) such that \(\vect x+\theta a\vect e_i\in U\) for all
  \(\theta\in[0,1]\). Then there is some \(\theta\in(0,1)\) such that
  \[
    f(\vect x+a\vect e_i)
      -
    f(\vect x)
      =
    a\frac{\partial f}{\partial x_i}(\vect x+\theta a\vect e_i)
  .\]
\end{lemma}
\begin{proof}

\end{proof}

\begin{proposition}[13.15, The Mean Value Proposition]
  Let \(\vect x\in\mb R^n\) and \(r>0\). Let \(f:B_r(\vect x)\to\mb R\)
  be a function with first-order partial derivatives. Then if
  \(\vect h\in\mb R^n\) satisfies \(\|\vect h\|<r\), then
  there are points \(\vect z_i\in B_{\|\vect h\|}(\vect x)\) for
  \(1\leq i\leq n\) satisfying
  \[
    f(\vect x+\vect h)
      -
    f(\vect x)
      =
    \sum_{i=1}^n
    p_i(\vect h)
    \frac{\partial f}{\partial x_i}(\vect z_i)
  .\]
\end{proposition}
\begin{proof}

\end{proof}

\begin{definition}
  Let \(\vect x\in U\subseteq\mb R^n\) where \(U\) is open, let
  \(f:U\to\mb R\) be a function, and let
  \(\vect p\in\mb R^n\setminus\{\vect 0\}\). Define the
  \textit{directional derivative} of \(f\) at \(\vect x\) in the direction
  \(\vect p\) by
  \[
    \frac{\partial f}{\partial \vect p}(\vect x)
      =
    \lim_{t\to 0}\frac{f(\vect x+t\vect p)-f(\vect x)}{t}
  \]
  whenever that limit exists.
\end{definition}

\begin{definition}
  Let \(\vect x\in U\subseteq\mb R^n\) where \(U\) is open, and let
  \(f:U\to\mb R\) be a function with first-order partial derivatives
  at \(\vect x\). Define its \textit{gradient} \(\nabla f(\vect x)\in\mb R^n\)
  at \(\vect x\) to satisfy \(p_i(\nabla f(\vect x))=f_{x_i}(\vect x)\)
  for all \(1\leq i\leq n\).
  If its gradient exists at every \(\vect x\in U\), then let
  \(\nabla f:U\to\mb R^n\) be the \textit{gradient function}
  defined by evaluating the gradient at each point.
\end{definition}

\begin{theorem}[13.16, The Directional Derivative Theorem]
  Let \(U\subseteq\mb R^n\) be open and \(f:U\to\mb R\) be continuously
  differentiable. Then for each \(\vect x\in U\) and
  \(\vect p\in\mb R^n\setminus\{\vect 0\}\), \(f\) has a directional derivative
  at \(\vect x\) in the direction of \(\vect p\) given by
  \[
    \frac{\partial f}{\partial \vect p}(\vect x)
      =
    \<\vect p,\nabla f(\vect x)\>
  .\]
\end{theorem}
\begin{proof}

\end{proof}

\begin{theorem}[13.17, The Mean Value Theorem]
  Let \(U\subseteq\mb R^n\) be open and \(f:U\to\mb R\) be continuously
  differentiable. Let \(\vect x,\vect h\in \mb R^n\)
  such that \(\vect x+\theta \vect h\in U\) for all
  \(\theta\in[0,1]\). Then there is some \(\theta\in(0,1)\) such that
  \[
    f(\vect x+\vect h)
      -
    f(\vect x)
      =
    \<\vect h,\nabla f(\vect x+\theta\vect h)\>
  .\]
\end{theorem}
\begin{proof}

\end{proof}

\begin{corollary}[13.18]
  Let \(\vect x\in U\subseteq\mb R^n\) with \(U\) open and let \(f:U\to\mb R\)
  be continuously differentiable such that
  \(\nabla f(\vect x)\not=\vect 0\). Then the unit vector maximizing
  the value of the directional derivative of \(f\) at \(\vect x\) is
  \[
    \vect p_0
      =
    \frac{\nabla f(\vect x)}{\|\nabla f(\vect x)\|}
  .\]
\end{corollary}
\begin{proof}

\end{proof}

\begin{theorem}[13.20]
  Let \(U\subseteq\mb R^n\) be open and \(f:U\to\mb R\) be continuously
  differentiable. Then \(f\) is continuous.
\end{theorem}
\begin{proof}

\end{proof}

\begin{exercise}[4]
  Suppose that \(f:\mb R^n\to\mb R\) has first-order partial derivatives
  and that \(\vect x\) is a local minimizer for \(f\), that is,
  there exists some \(\epsilon>0\) such that for all
  \(\vect y\in B_\epsilon(\vect x)\), \(f(\vect y)\geq f(\vect x)\).
  Prove that \(\nabla f(\vect x)=\vect 0\).
\end{exercise}
\begin{solution}

\end{solution}


\setcounter{chapter}{-1}
\chapter[Log, Exp, and Trig Functions]{Logrithmic, Exponential, and Trigonometric Functions}

Assume all sequences in this chapter are indexed by the nonnegative integers,
and that \(0^0=1\).

\section{Logrithms and Exponential Functions}

\begin{theorem}[The Identity Criterion]
  A differentiable function \(g:U\to\mb R\) where \(U\) is an open interval
  of \(\mb R\) is the constant function \(0\) if and only if
  \[
    \begin{cases}
      g'(x)=0 & \text{ for all } x\in U \\
      g(x_0)=0 & \text{ for some } x_0\in U
    \end{cases}
  .\]
\end{theorem}
\begin{proof}
  Proved in MATH 3141.
\end{proof}

\begin{proposition}[7.1]
  Let \(x_0\in U\) for some open interval \(U\subseteq\mb R\),
  and suppose \(f:(a,b)\to\mb R\) is continuous.
  Then for any number \(y_0\), the function \(F:U\to\mb R\) defined by
  \[
    F(x) = y_0 + \int_{x_0}^x f
  \]
  is the unique funciton satisfying
  the conditions of the following differential equation:
  \[
    \begin{cases}
      F'(x)=f(x) & \text{ for all } x\in U \\
      F(x_0)=y_0
    \end{cases}
  .\]
\end{proposition}
\begin{proof}

\end{proof}

\begin{definition}
  Define the \textit{natural logrithm} function
  \(\ln:(0,\infty)\to\mb R\) by \(\ln(x)=\int_1^x\frac{1}{t}~dt\).
\end{definition}

\begin{theorem}[5.1]
  The natural logrithm function satisfies the following properties for
  all \(a,b>0\), \(c\in\mb R\), and \(r\in\mb Q\).
  \begin{enumerate}[(a)]
    \item \(\ln(ab)=\ln(a)+\ln(b)\).
    \item \(\ln(a^r)=r\ln(a)\).
    \item There exists \(x>0\) such that \(\ln(x)=c\).
  \end{enumerate}
\end{theorem}
\begin{proof}
  \begin{enumerate}[(a)]
    \item

    \item

    \item

  \end{enumerate}
\end{proof}

\begin{proposition}[pg. 120]
  The natural logrithm
  \(\ln:(0,\infty)\to\mb R\)
  has a differentiable inverse function \(\exp:\mb R\to(0,\infty)\),
  called the \textit{exponential} function.
\end{proposition}
\begin{proof}

\end{proof}

\begin{definition}
  Let the \textit{natural number}
  \(e\in\mb R\) be defined by \(\exp(1)\). (Note
  \(\ln(e)=\ln(\exp(1))=1\).)
\end{definition}

\begin{proposition}[pg. 120]
  The exponential function \(\exp\) is the unique solution to the
  differential equation
  \[
    \begin{cases}
      \frac{d}{dx}[\exp(x)]=\exp(x) & \text{ for all } x\in \mb R \\
      \exp(0)=1
    \end{cases}
  .\]
\end{proposition}
\begin{proof}

\end{proof}

\begin{definition}
  Define \(a^x=\exp(x\ln(a))\) for \(a>0\) and all real numbers \(x\).
  (In particular, \(e^x=\exp(x\ln(e))=\exp(x)\).)
\end{definition}

\begin{proposition}[pg. 121]
  The above definition is compatible with the previously given definition
  for exponentiation for rational numbers.
\end{proposition}
\begin{proof}

\end{proof}

\begin{proposition}[5.2, 5.3]
  Let \(a>0\) and \(r\in\mb R\). Then
  \[
    \frac{d}{dx}[a^x]=a^x\ln a
    ~~~\text{ for all } x\in\mb R \text{, and}
  \]
  \[
    \frac{d}{dx}[x^r]=rx^{r-1}
    ~~~\text{ for all } x>0
  .\]
\end{proposition}
\begin{proof}

\end{proof}

\begin{theorem}[5.4]
  Let \(c,k\in\mb R\). Then \(F(x)=ce^{kx}\) is the unique solution
  to the differential equation
  \[
    \begin{cases}
      F'(x)=kF(x) & \text{ for all } x\in \mb R \\
      F(0)=c
    \end{cases}
  .\]
\end{theorem}
\begin{proof}

\end{proof}

\section{Power Series}

\begin{definition}
  For a sequence of real numbers \(\{a_k\}\), let \(\{s_k\}\) be its
  \textit{sequence of partial sums} defined by
  \[
    s_n = \sum_{k=0}^{n} a_k
  .\]
\end{definition}

\begin{definition}
  Define the \textit{series} of a sequence \(\{a_k\}\) by
  \[
    \sum_{k=0}^\infty a_k
      =
    \lim_{n\to\infty} s_n
      =
    \lim_{n\to\infty}\left[\sum_{k=0}^{n} a_k\right]
  \]
  whenever the limit exists. If so, the series is said to \textit{converge},
  and if not, the series is said to \textit{diverge}.
\end{definition}

\begin{definition}
  Let \(\{c_k\}\) be a sequence of real numbers, and let
  \[
    D
      =
    \{
      x\in\mb R
    :
      \sum_{k=0}^\infty c_kx^k
      \text{ converges}
    \}
  .\]
  Let \(f:D\to\mb R\) defined by \(f(x)=\sum_{k=0}^\infty c_kx^k\) be
  the \textit{power series expansion} of \(\{c_k\}\), and call
  \(D\) its \textit{domain of convergence}.
\end{definition}

In the interest of time, we will assume all given series converge on
an appropriate domain (you did many such problems in your
Calculus II course), as well as assume the following theorem without proof.

\begin{theorem}[9.41]
  Let \(D\) be the domain of convergence of the power series expansion
  of \(\{c_k\}\), and suppose \(r>0\) satisfies \((-r,r)\subseteq D\).
  Then \(f:(-r,r)\to\mb R\) defined by
  \(f(x)=\sum_{k=1}^\infty c_kx^k\) has derivatives of all orders
  satisfying
  \[
    \frac{d^n}{dx^n}[f(x)]
      =
    \sum_{k=0}^\infty \frac{d^n}{dx^n}[c_kx^k]
  .\]
\end{theorem}

\begin{theorem}
  \[
   \exp(x)
      =
    \sum_{k=0}^\infty \frac{1}{k!}x^k
  .\]
\end{theorem}
\begin{proof}

\end{proof}

\section{Trigonometric Functions}

\begin{definition}
  Define the \textit{cosine} function \(\cos:\mb R\to\mb R\) by
  \[
    \cos(x)
      =
    \sum_{k=0}^\infty
    \frac{(-1)^k}{(2k)!}x^{2k}
  .\]
\end{definition}

\begin{theorem}[9.42]
  The cosine function is a solution to the differential equation
  \[
    \begin{cases}
      F''(x)=-F(x) & \text{ for all } x\in \mb R \\
      F(0)=1 \\
      F'(0)=0
    \end{cases}
  .\]
\end{theorem}
\begin{proof}

\end{proof}

\begin{lemma}[5.5]
  The unique solution to the differential equation
  \[
    \begin{cases}
      z''(x)=-z(x) & \text{ for all } x\in \mb R \\
      z(0)=0 \\
      z'(0)=0
    \end{cases}
  \]
  is the constant function valued at \(0\).
\end{lemma}
\begin{proof}

\end{proof}


\begin{corollary}
  \(\cos(x)\) is the unique solution to the differential equation
  \[
    \begin{cases}
      F''(x)=-F(x) & \text{ for all } x\in \mb R \\
      F(0)=1 \\
      F'(0)=0
    \end{cases}
  .\]
\end{corollary}
\begin{proof}

\end{proof}

\begin{definition}
  Let the \textit{sine} function
  \(\sin:\mb R\to\mb R\) be defined by \(\sin(x)=-\frac{d}{dx}[\cos x]\).
\end{definition}

\begin{theorem}
  The sine function satisfies
  \[
    \sin(x)
      =
    \sum_{k=0}^\infty
    \frac{(-1)^k}{(2k+1)!}x^{2k+1}
  \]
  and is the unique solution to the differential equation
  \[
    \begin{cases}
      F''(x)=-F(x) & \text{ for all } x\in \mb R \\
      F(0)=0 \\
      F'(0)=1
    \end{cases}
  .\]
\end{theorem}
\begin{proof}

\end{proof}

\begin{theorem}[5.6]
  For all real numbers \(x,y\):
  \begin{enumerate}[(a)]
    \item \([\sin x]^2+[\cos x]^2=1\).
    \item \(\sin(x+y)=\sin(x)\cos(y)+\cos(x)\sin(y)\).
    \item \(\cos(x+y)=\cos(x)\cos(y)-\sin(x)\sin(y)\).
    \item \(|\sin x|\leq 1\) and \(|\cos x|\leq 1\).
  \end{enumerate}
\end{theorem}

\begin{definition}
  A function \(f:\mb R\to\mb R\) is said to be \textit{periodic} with
  a \textit{period} \(T>0\) if it satisfies
  \[
    f(x+T)=f(x)
    ~~~\text{for all }x\in\mb R
  .\]
\end{definition}

\begin{theorem}[5.7]
  There exists a smallest \(x>0\) satisfying \(\cos(x)=0\).
\end{theorem}
\begin{proof}

\end{proof}

\begin{definition}
  Define the number \(\pi\in\mb R\) such that \(x=\pi/2\) is the
  smallest positive number satisfying \(\cos(x)=0\).
\end{definition}

\begin{theorem}[5.8]
  The sine and cosine functions are both periodic with period \(2\pi\).
\end{theorem}
\begin{proof}

\end{proof}

\begin{theorem}[7.7, Integration by Substitution]
  Let \(f:[a,b]\to\mb R\) and \(g:[c,d]\to\mb R\) be continuous, such that
  \(g:(c,d)\to\mb R\) has a bounded continuous derivative and
  \(g[(c,d)]\subseteq(a,b)\). Then
  \[
    \int_c^d (f\circ g)g'
      =
    \int_{g(c)}^{g(d)} f
  .\]
\end{theorem}
\begin{proof}

\end{proof}

\begin{proposition}[7.8, The Geometric Interpretation of \(\pi\)]
  \[
    \int_0^1\sqrt{1-x^2}dx = \frac{\pi}{4}
  .\]
  That is, \(\pi\) is the area of a circle with radius \(1\).
\end{proposition}
\begin{proof}

\end{proof}











\chapter*{Midterm Part 3}

Choose two of the below problems (which you did not choose for
Part 2) and typeset your solutions. Delete the other three.
Each will be worth 20/100 points towards your midterm grade
for a total of 40/100 points.


\begin{exercise}[1]
Prove that if \(Q_n\) is a partition of \([a,b]\) refining
the partition \(P_n\) of \([a,b]\) for each
natural number \(n\), and \(\{P_n\}\) is an Archimedian sequence of
partitions for \(f\) on \([a,b]\), then \(\{Q_n\}\) is also Archimedian.
\end{exercise}
\begin{solution}

\end{solution}

\begin{exercise}[2]
Explain the error(s) in the following ``proof'',
and then give a counterexample showing that the theorem is false.

\textbf{Theorem:} If \(f:[0,1]\to\mathbb R\) is integrable, then
\(f\) is also continuous.

\textbf{Proof:}
Since \(f\) is integrable, we may define \(F:[0,1]\to\mathbb R\)
by \(F(x)=\int_0^x f\). It follows
that \(F(x)\) is a differentiable function, because it is an antiderivative
of \(f\). Thus \(\frac{d}{dx}[F(x)]=f(x)\) by the Second Fundamental
Theorem of Calculus. Since the derivative of any differentiable function
is continuous, we conclude \(f\) is continuous.
\end{exercise}
\begin{solution}

\end{solution}

\begin{exercise}[3]
Recall that an \textbf{even} function satisfies the condition \(f(x)=f(-x)\).
Let \(f:\mathbb R\to\mathbb R\) be an even continuous function.
Prove that
\[
  \frac{d}{dx}\left[\int_{-x}^x f\right]=2f(x)
.\]

(Hint: Corollary 6.30 says that \(\frac{d}{dx}[\int_{x}^0f]=-f(x)\).)
\end{exercise}
\begin{solution}

\end{solution}

\begin{exercise}[4]
Prove the following theorem:

Let \(\vect x\in\mathbb R^n\) and let \(\{\vect x_k\}\) be a sequence
of points in \(\mathbb R^n\). If for every open set \(U\)
containing \(\vect x\), there is an index \(K\) such that
\(\vect x_k\in U\) for all \(k\geq K\),
then \(\{\vect x_k\}\) converges to \(\vect x\).

(Hint: \(B_\epsilon(\vect x)\) is open.)
\end{exercise}
\begin{solution}

\end{solution}

\begin{exercise}[5]
Prove that any finite subset of \(\mathbb R^n\) is closed.

(Hint: First
prove that any singleton subset of \(\mathbb R^n\) is closed.)
\end{exercise}
\begin{solution}

\end{solution}






\chapter*{Final Exam Part 3}

Delete any exercises you submitted for Part 2, and choose two
others to delete as well. Typeset the solutions to the remaining
exercises (so that you will have submitted six proofs in total
for the final exam).
Each will be worth 15/100 points towards your final exam grade.


\begin{exercise}[1]
Define \(f:[0,2]\to\mathbb R\) by \(f(x)=3x\). Explicitly define a sequence of
partitions \(\{P_n\}\) of \([0,2]\), and then prove that this sequence
is Archimedian for \(f\) on \([0,2]\).
\end{exercise}
\begin{solution}

\end{solution}

\begin{exercise}[2]
Compute the boundary of the subset \(\mathbb Q^2\) of \(\mathbb R^2\).
\end{exercise}
\begin{solution}

\end{solution}

\begin{exercise}[3]
Prove that the function \(f:\mathbb R^2\to\mathbb R\) defined by
\[
f(x,y) =
\begin{cases}
  \frac{x^2y}{x^3+y^3} & (x,y)\not=(0,0) \\
  0 & (x,y)=(0,0)
\end{cases}
\]
is not continuous.
\end{exercise}
\begin{solution}

\end{solution}

\begin{exercise}[4]
Let \(f:\mathbb R^n\to\mathbb R\) be defined by
\[
f(\vect x) =
\begin{cases}
  \|\vect x\| & \|\vect x\|\in \mathbb R\setminus\mathbb Q\\
  0 & \|\vect x\|\in\mathbb Q
\end{cases}
.\]
Prove that \(f\) is continuous at \(\vect 0\).
\end{exercise}
\begin{solution}

\end{solution}

\begin{exercise}[5]
Let \(d:X^2\to[0,\infty)\) be a metric on \(X\). Prove that the function
\(e:X^2\to[0,\infty)\) defined by \(e(x,y)=\min(d(x,y),1)\) satisfies
the triangle inequality for the below cases:
\end{exercise}
\begin{solution}
\textbf{Case \(e(x,y)+e(y,z)<1\):}

\textbf{Case \(e(x,y)+e(y,z)\geq 1\):}

\end{solution}


\begin{exercise}[6]
Compute \(f_y(0,0)\) where \(f:\mathbb R^2\to\mathbb R\) is defined by
\[
f(x,y) =
\begin{cases}
  \frac{x^2y}{x^3+y^3} & (x,y)\not=(0,0) \\
  0 & (x,y)=(0,0)
\end{cases}
.\]
\end{exercise}
\begin{solution}

\end{solution}

\begin{exercise}[7]
Let \(f:\mathbb R^n\to\mathbb R\) be continuous such that
\(f(\vect 0)=0\). Prove that for each vector \(\vect p\in\mathbb R^n\),
there exists another vector \(\vect x\in\mathbb R^n\) such that
\(\frac{\partial f}{\partial\vect p}(\vect x)=f(\vect p)\).
\end{exercise}
\begin{solution}

\end{solution}

\begin{exercise}[8]
Recall that \(\ln:(0,\infty)\to\mathbb R\) is defined
by \(\ln(x)=\int_1^x\frac{1}{t}dt\).
Show that \(f:(-\infty,0)\to\mathbb R\) defined by
\(f(x)=\ln(-x)\) is the unique solution to the differential equation
\[
\begin{cases}
  f'(x)=\frac{1}{x} & \forall x\in(-\infty,0) \\
  f(-1)=0
\end{cases}
.\]
\end{exercise}
\begin{solution}

\end{solution}


\end{document}


